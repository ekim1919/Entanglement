\documentclass[twoside]{article}
\setlength{\oddsidemargin}{0.25 in}
\setlength{\evensidemargin}{-0.25 in}
\setlength{\topmargin}{-0.6 in}
\setlength{\textwidth}{6.5 in}
\setlength{\textheight}{8.5 in}
\setlength{\headsep}{0.75 in}
\setlength{\parindent}{0 in}
\setlength{\parskip}{0.1 in}

%
% ADD PACKAGES here:
%

\usepackage{amsfonts}
\usepackage{amsmath}
\usepackage{amssymb}

\usepackage{geometry}
\usepackage{graphicx}
\usepackage{physics}

\usepackage{times}
\usepackage{microtype}

\usepackage{stmaryrd}
\usepackage{cancel}
\usepackage{mdframed}

\usepackage{mathtools}
\usepackage{flexisym}
\usepackage{hyperref} % backref=page
\usepackage[shortlabels]{enumitem}

% Prints a trailing space in a smart way.
\usepackage{xspace}
\usepackage[font=footnotesize]{caption}
\usepackage{titlesec}


\DeclarePairedDelimiter\ceil{\lceil}{\rceil}
\DeclarePairedDelimiter\floor{\lfloor}{\rfloor}
\newenvironment{proof}[1][Proof]{\textbf{#1.} }{\ \rule{0.5em}{0.5em}}


% General Mathematical Notation
\newcommand{\reals}{{\mathbb{R}}}
\newcommand{\complex}{{\mathbb{C}}}                    %reals                  %reals
\newcommand{\nnreal}{{\nnnum R}}
                   %nonnegative

\newcommand{\nat}{\mathbb{N}}                      %natural numbers
\newcommand{\integers}{{\num Z}}                      %integers
\newcommand{\rat}{{\num Q}}                      %rationals
\newcommand{\nnrat}{{\nnnum Q}}

\newcommand{\boldu}{\textbf{u}}
\newcommand{\boldv}{\textbf{v}}
%\newcommand{\ip}[1][2]{\langle {#1}, {#2},\rangle}
% Environments
\newtheorem{theorem}{Theorem}[section]
\newtheorem{lemma}{Lemma}[section]
\newtheorem{proposition}[theorem]{Proposition}
\newtheorem{claim}[theorem]{Claim}
\newtheorem{corollary}[theorem]{Corollary}
\newmdtheoremenv{definition}[theorem]{Definition}
% \newenvironment{proof}{{\bf Proof:}}{\hfill\rule{2mm}{2mm}}
% Vector Spaces

% Hilbert Space
\newcommand{\hil}{\mathcal{H}}
\newcommand{\bhilop}{\mathcal{B}(\mathcal{H})}

% Differentiation
\newcommand{\totder}[1][]{\frac{d^{#1}}{dx^{#1}}}

% Integration
\newcommand{\doubinfint}{\int_{-\infty}^{\infty}}

% Series
\newcommand{\powaseries}[1][0]{\sum_{n = {#1}}^{\infty}}

% Distributions
\newcommand{\testfunc}{C_F^{\infty}(\reals)}
\newcommand{\prinval}[1]{P\left({#1}\right)}

% Typesetting
\newcommand{\para}[1]{\left( {#1} \right)}
% \newcommand{\brack}[1]{\left[ {#1} \right

% Coding Theory
\newcommand{\code}{\mathcal{C}}
\newcommand{\tanner}{\mathcal{T}(V,C,E)}


\geometry{letterpaper,left=1in,top=1in,right=1in,bottom=1in, footskip=0.5in, nohead}


%
% The following commands set up the lecnum (lecture number)
% counter and make various numbering schemes work relative
% to the lecture number.
%
\newcounter{lecnum}
\renewcommand{\thepage}{\thelecnum-\arabic{page}}
\renewcommand{\thesection}{\thelecnum.\arabic{section}}
\renewcommand{\theequation}{\thelecnum.\arabic{equation}}
\renewcommand{\thefigure}{\thelecnum.\arabic{figure}}
\renewcommand{\thetable}{\thelecnum.\arabic{table}}

%
% The following macro is used to generate the header.
%

\newcommand{\seasonyear}{Fall 2022}

\newcommand{\lecture}[3]{
   \pagestyle{myheadings}
   \thispagestyle{plain}
   \newpage
   \setcounter{lecnum}{#1}
   \setcounter{page}{1}
   \noindent
   \begin{center}
   \framebox{
      \vbox{\vspace{2mm}
    \hbox to 6.28in { {\bf Directed Reseach
	\hfill } {\bf \seasonyear}}
       \vspace{4mm}
       \hbox to 6.28in { {\Large \hfill QLDPC Codes Part #1: #2  \hfill} }
       \vspace{2mm}
       \hbox to 6.28in { {\it Scribe: #3 \hfill} }
      \vspace{2mm}}
   }
   \end{center}
}
%
% Convention for citations is authors' initials followed by the year.
% For example, to cite a paper by Leighton and Maggs you would type
% \cite{LM89}, and to cite a paper by Strassen you would type \cite{S69}.
% (To avoid bibliography problems, for now we redefine the \cite command.)
% Also commands that create a suitable format for the reference list.
\renewcommand{\cite}[1]{[#1]}
\def\beginrefs{\begin{list}%
        {[\arabic{equation}]}{\usecounter{equation}
         \setlength{\leftmargin}{2.0truecm}\setlength{\labelsep}{0.4truecm}%
         \setlength{\labelwidth}{1.6truecm}}}
\def\endrefs{\end{list}}
\def\bibentry#1{\item[\hbox{[#1]}]}

%Use this command for a figure; it puts a figure in wherever you want it.
%usage: \fig{NUMBER}{SPACE-IN-INCHES}{CAPTION}
\newcommand{\fig}[3]{
			\vspace{#2}
			\begin{center}
			Figure \thelecnum.#1:~#3
			\end{center}
	}

% **** IF YOU WANT TO DEFINE ADDITIONAL MACROS FOR YOURSELF, PUT THEM HERE:

\newcommand\E{\mathbb{E}}

\begin{document}
%FILL IN THE RIGHT INFO.
%\lecture{**LECTURE-NUMBER**}{**DATE**}{**LECTURER**}{**SCRIBE**}
\lecture{1}{The Quantum Hypergraph Product}{Edward Kim}
\section{Kiteav's Toric Code}

Let us first very briefly recall a ``vanilla" version of Kiteav's 2D toric code. The toric code is typically defined on an $m \times m$ lattice with the data qubits placed along the edges of the lattice. In total, there will be $2m^2$ qubits. The boundary is set to be \emph{periodic}: pairs of vertices of the opposing ends each row and column are identified with each other.

Under this identification, each vertex would be located at the intersection of precisely four edges. For each such vertex $v$, identify $e_{1},e_v,e_3, e_4$ as the four edges indicent to $v$ on the lattice. Define the $X$-stabilizers $X_v = X_{e_1}X_{e_2}X_{e_3}X_{e_4}$ to have support on precisely the qubits located on the edges neighboring $v$.

Similarly, every $4$-cycle determines a \emph{face} or \emph{plaquette} $f$ on the lattice. For each such plaquette, identify $e'_{1},e'_2,e'_3, e'_4$ define $Z_f = Z_{e'_1}Z_{e'_2}Z_{e'_3}Z_{e'_4}$ to be the $Z$-stabilizer having support on the precisely the qubits associated to the each edge in the $4$-cycle.

Altogether there will be $2m^2$ qubits and $2m^2 - 2$ independent stabilizers as suggested by the dependencies:
\[ \prod_{f \in F} Z_f = \prod_{v \in V} X_v = I \]

Thus, the toric code as presented here will turn out to be a $\llbracket 2m^2, 2, m \rrbracket$ quantum CSS code. It is well-known that such a code can be constructed by cellulating the three-dimensional torus $\mathbb{T}^2$.

\section{Shortcomings}

Observe that although the relative distance scales $O(\sqrt{N})$ where $N = 2m^2$ is the \emph{blocklength} of the code, the encoding rate vanishes as $N \rightarrow \infty$. In particular, as we increase the dimensions of our lattice, we can still only encode two logical qubits.

The reader is likely to be aware that several variants of the toric code which allow a higher number of logical qubits along with clever topological schemes to implement Clifford gates, state injection, etc have been fervently investigated. However, as rich and intriguing these schemes are, we will not be following these topological approaches.
%
Instead, we will attempt to first understand the toric code through a more \emph{graph-theoretic} perspective.

The starting idea behind the construction of the \textbf{Quantum Hypergraph Product (QHP)} is that the toric code can be seen as a type of \emph{graph product} between two simpler codes, namely two copies of a repetition code.
%
The astute insight arises from the subsequent observation that this graph product can be generalized to two \emph{arbitrary} input classical linear codes by taking the product graph of the Tanner graphs associated to the two linear codes and designating two particular subgraphs to be the linear codes $\mathcal{C}_X, \mathcal{C}_Z$ comprising our quantum CSS code $\mathcal{Q}(\mathcal{C}_X, \mathcal{C}_Z)$.

To this end, let us describe some basic notions from which this graph product surfaces rather naturally.

\section{Tanner Graphs}

Fix a classical binary linear code $\code$ with parameters $[n,k,d]$, and  denote $H(\code)$ as its $(n-k) \times n$ parity check matrix. We can define a biparitite graph $\tanner = (V \cup C, E)$ by placing the code-bits on the left partition $V$ and the parity checks on the right partition. An edge between a code-bit vertex $v \in V$ and a parity check node $c \in C$ exists when $v$ participates in the check $c$. The precise definition will be formulated as follows:

\begin{definition}
  Let $H = H(\code)$ be a $(n-k) \times n$ parity check matrix. The \emph{Tanner Graph} $\tanner = (V \cup C, E)$ is an undirected bipartite graph such that $|V| = n$ and $|C| = n-k$. The elements of the left partition $V$ will be deemed as \emph{vertices} and the elements of the right partition $C$ will be deemed as \emph{check nodes}. An edge $(v,c) \in E$ between a vertex and check node if and only if $H_{cv} = 1$.
\end{definition}

Figure \ref{fig:TanGraphHamming} depicts the Tanner Graph for the $[7,4,3]$ Hamming Code.

%Subfigur for the repetition code.

\begin{figure}[t!]
  \centering
  \includegraphics[width=3in]{example-image-a}
  \caption{The Tanner Graph for the $[7,4,3]$ Hamming Code}
  \label{fig:TanGraphHamming}
\end{figure}


 \end{document}
